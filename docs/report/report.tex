\documentclass{article}
\usepackage{graphicx} % Required for inserting images

\title{ESIOT 24/25 Assignment \#3 - Report}
\author{Matte Bagnolini}
\date{February 2025}

\begin{document}

\maketitle

\section{Introduction}
The smart temperature system is made up of four main subsystems:
\begin{enumerate}
    \item \textbf{Window Controller}: it controls an Arduino board, managing the opening of the window.
    \item \textbf{Temperature Monitoring Subsystem}: it manages an ESP32 board to monitor the temperature through a temperature sensor.
    \item \textbf{Control Unit Back-End}: the core of the system, managing the communication between the subsystems
    \item \textbf{Dashboard Front-End}: a web application dashboard that allows operators to interface with the system
\end{enumerate}

\newpage

\section{Window Controller}
The window controller is an Arduino-based application designed and implemented using synchronous Finite State Machines and a task-based architecture.\\
It consists of two tasks: 
\begin{enumerate}
    \item \textbf{Window Controlling Task}: responsible for controlling the opening of the window, implemented as an FSM shown in \textbf{Figure \ref{fig:window controlling task}}
    \item \textbf{Communications Task}: responsible for handling the serial line communication with the backend
\end{enumerate}
\begin{figure}[htp]
    \centering
    \includegraphics[width=5cm]{WindowControllingTask.png}
    \caption{Window Controlling Task State diagram }
    \label{fig:window controlling task}
\end{figure}

\section{Temperature Monitoring Subsystem}
The temperature monitoring subsystem is an ESP32-based application designed and implemented using synchronous Finite State Machines and a task-based architecture.\\
It is made up of two components:
\begin{enumerate}
    \item \textbf{Temperature Task}: to measure the temperature, implemented as a FSM as shown in \textbf{Figure \ref{fig:temp task}} 
    \item \textbf{Communications Component}: used by the temperature task to transmit sampled data to the backend unit via the MQTT protocol, using \textbf{broker.mqtt-dashboard.com} as the broker
\end{enumerate}
\begin{figure}[htp]
    \centering
    \includegraphics[width=5cm]{TemperatureTask.png}
    \caption{Temperature Task State diagram }
    \label{fig:temp task}
\end{figure}
\newpage

\section{Control Unit Back-End}
The Control Unit Back-End is an application written in Golang responsible for managing and coordinating the execution of all the subsystems. It is designed and implemented as a Finite State Machine, as illustrated in \textbf{Figure \ref{fig:backend}}\\
\begin{figure}[htp]
    \centering
    \includegraphics[width=9cm]{backend.png}
    \caption{Control Unit State Diagram}
    \label{fig:backend}
\end{figure}
\newpage
The Control Unit Back-End implementation is divided into packages:
\begin{itemize}
    \item \textbf{serial}: manages serial line connection and communications with Arduino
    \item \textbf{mqtt}: manages mqtt connection and communications with ESP32
    \item \textbf{http}: manages http connection and communications with the dashboard
    \item \textbf{models}: implements the logic of the system, including the state machine
    \item \textbf{db}: manages the database to store the historical data
\end{itemize}
\subsection{Temperature Database}
The subsystem uses a simple database to store historical data. In this way, it is possible to retain data even if the back-end suddenly crashes.\\
It is implemented as a \textbf{SQLite} database, which is simple to set up but not so reliable if the system needs to scale. However, since we can expect this system to be used by only one user at a time, SQLite is an appropriate choice.\\
\\
The database consists of a single table, containing average, minimum, and maximum temperatures sampled on a specific date. The table is shown in \textbf{Figure \ref{fig:db}}.
\begin{figure}[htp]
    \centering
    \includegraphics[width=4cm]{db.png}
    \caption{Database Structure}
    \label{fig:db}
\end{figure}

\section{Dashboard Front-End}
The Dashboard Front-End is a web app that interacts with the back-end, developed with \textbf{HTML} and \textbf{JavaScript}.\\
It shows some information about the system and allows to resolve the alarm and manually select the opening of the window. It also displays a graphic of the last measurements, and shows the historical temperature data stored in the backend database.\\
\textbf{Figure \ref{fig:dashboard}} and \textbf{Figure \ref{fig:history}} show the dashboard front-end. 
\begin{figure}[htp]
    \centering
    \includegraphics[width=9cm]{dashboard.png}
    \caption{Dashboard}
    \label{fig:dashboard}
\end{figure}
\begin{figure}[htp]
    \centering
    \includegraphics[width=9cm]{history.png}
    \caption{History}
    \label{fig:history}
\end{figure}
\\
\section{Circuits Board}
\textbf{Figure \ref{fig:arduino}} shows the Arduino board, while \textbf{Figure \ref{fig:esp32}} shows the ESP32 board.
\begin{figure}[htp]
    \centering
    \includegraphics[width=12cm]{arduino.png}
    \caption{Dashboard}
    \label{fig:arduino}
\end{figure}
\begin{figure}[htp]
    \centering
    \includegraphics[width=12cm]{esp32.png}
    \caption{History}
    \label{fig:esp32}
\end{figure}

\end{document}

